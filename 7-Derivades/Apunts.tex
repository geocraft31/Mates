\documentclass[12pt,a4paper]{article}
\usepackage[utf8]{inputenc}
\usepackage[left=2cm,right=2cm,top=2cm,bottom=2cm]{geometry}
\usepackage{amsfonts}
\usepackage{amsmath}
\usepackage{enumitem}
\usepackage{tikz}
\usepackage{pgfplots}

\newcommand{\set}[1]{\mathbb{#1}}
\newcommand{\pvec}[2]{\begin{pmatrix}#1\\#2\end{pmatrix}}
\newcommand{\module}[1]{\|#1\|}

\title{Derivades}
\author{}
\date{2022 - 2023}

\pgfplotsset{compat=1.18}

\begin{document}

\maketitle

\section{Taxa de variació mitjana (TVM)}

La taxa de variació mitjana d'una funció, és com varien els valors de $y$ respecte dos punts.\\
La TVM és el pendent de la recta que pasa pels dos punts d'una funció, així:
$$TVM=\dfrac{f(a+h)-f(a)}{a+h-a}$$

\section{Derivdada en un punt}
La derivada en un punt és la taxa de variació instantània, es a dir, és el pendent de la recta tangent de la funció en aquell punt.
$$\lim_{h\to0}\dfrac{f(a+h)-f(a)}{h}$$
Note:\\

Si $f$ és la funció inical, $f'$ és la derivada en mates i $\dfrac{df}{dx}$ és la derivada en física.
\section{Derivada d'una funció}
La derivada d'una funció és la funció que assigna a cada $x$ el valor de la seva derivada.\\
Per trobar la derivada d'una funció pots aplicar la formula de la taxa de variació instantània per a $h \to 0$.
\subsection{Derivada d'una poténcia, $f(x)=x^n$}
Per combartir una funció d'estil $f(x)=x^2$ a la seva derivada, podem aplicar la següent formula:
$$f(x)=x^n\rightarrow f'(x)=n\cdot x^{n-1}$$
Així $f(x)=x^2 \to f'(x)=2\cdot x^{2-1} \rightarrow f'(x)=2\cdot x$ 

\end{document}

