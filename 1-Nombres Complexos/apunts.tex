\documentclass[12pt,a4paper]{article}
\usepackage[left=2cm,right=2cm,top=2cm,bottom=2cm]{geometry}
\usepackage{amsfonts}
\usepackage{amsmath}
\usepackage{enumitem}
\usepackage{tikz}
\usepackage{pgfplots}

\newcommand{\reals}{\mathbb{R}}
\newcommand{\module}[1]{\|#1\|}
\newcommand{\complex}{\mathbb{C}}
\newcommand{\drawgraph}{ \draw[-] (-1, 0) -- (1, 0) node[right] {$x$};
\draw[-] (0, -1) -- (0, 1) node[above] {$y$};}

\title{Equacions i Inequacions}
\author{-}
\date{Febrer 2023}

\pgfplotsset{compat=1.18}
\begin{document}

\maketitle
\section{Unitat imaginaria}
$i$ és la unitat imaginaria i és el resultat de $\sqrt{-1}$
\subsection{Nombre Complex}
És un numero de la forma $a+bi$, on $a$, $b \in \reals$, $a$ és la part real i $b$ la part imaginária
\subsection{Nombre Conjugat}
Si $Z=a+bi$, el seu conjugat $\overline{Z}=a-bi$
\subsection{Model i argument d'un $\complex$}
Sigui $Z=a+bi$
$$\module{Z}=\sqrt{a^2+b^2}$$
$$arg(Z)=\arctan\left(\frac{b}{a}\right)$$
Cal mirar sempre a quin quadrant es troba $Z$
\section{Representació de nombre complexos}
\subsection{Binòmica}
$$a+bi; a,b \in \reals$$
\subsection{Vectorial}
$$\begin{pmatrix}
    a\\
    b
\end{pmatrix}$$
\subsection{Trigonomètrica}
modul = $r$ i argument = $\alpha$
$$r\left(\cos\left(\alpha\right)+i\cdot\sin\left(\alpha\right)\right)$$
\subsection{Polar}
$$r_{\alpha}$$
\subsection{Exponencial}
$$r\cdot e^{i\cdot\alpha}$$
\subsection{Cani de representació}
De binòmica o vectorial a polar, exponencial o trigonomètrica.
$$r=\sqrt{a^2+b^2}$$
$$\alpha=\arctan\left(\frac{b}{a}\right)$$
$\alpha$ depen del quadrant.\\
De trigonomètrica, polar o exponencial a binòmica o vectorial.
$$a=r\cdot\cos(\alpha)$$
$$b=r\cdot\sin(\alpha)$$
\section{Operacions amb complexos en forma binòmica}
Siguin $Z_1=a_1+b_1i$ i $Z_2=a_2+b_2i$
\subsection{Suma i resta}
$$Z_1+Z_2=\left(a_1+a_2\right)+\left(b_1+b_2\right)\cdot i$$
$$Z_1-Z_2=\left(a_1-a_2\right)+\left(b_1-b_2\right)\cdot i$$
\subsection{Producte de complexos}
$$Z_1\cdot Z_2=\left(a_1+b_1i\right)\cdot\left(a_2+b_2i\right)=a_1a_2+a_1b_2i+a_2b_1i+b_1b_2i^2$$
Recorda $i^2=-1$
\subsection{Quaccient de complexos}
$$\frac{Z_1}{Z_2}=\frac{a_1a_2+b_1b_2}{a_2^2+b_2^2}+\frac{b_1a_2-a_1b_2}{a_2^2+b_2^2}i$$
\subsection{Potencies de i}
$i=\sqrt{-1}$\\
$i^2=\sqrt{-1}^2=-1$\\
$i^3=i^2\cdot i =-\sqrt{-1}$\\
$i^4=i^2\cdot i^2 = (-1)\cdot(-1)=1$\\
$i^5=i$
\section{Operacions amb complexos de forma polar}
Sigui $Z_1=\left(r_1\right)_{\alpha_1}$ i $Z_2=\left(r_2\right)_{\alpha_2}$
\subsection{Mutiplicació}
$$Z_1\cdot Z_2 = \left(r_1 \cdot r_2\right)_{\alpha_1+\alpha_2}$$
\subsection{Divisió}
$$\frac{Z_1}{Z_2}=\left(\frac{r_1}{r_2}\right)_{\alpha_1-\alpha_2}$$
\subsection{Potència}
$$\left(Z_1\right)^n=\left(r_1^n\right)_{n\cdot\alpha_1}$$
\subsection{Arrels amb complexos}
$$\sqrt[n]{Z_1}=\sqrt[n]{r}_{\frac{\alpha}{n}+\frac{2\pi\cdot k}{n}}$$
Recorda que $k = 0,1,2\dots n-1$.
\end{document}
