\documentclass[12pt,a4paper]{article}
\usepackage[left=2cm,right=2cm,top=2cm,bottom=2cm]{geometry}
\usepackage{gensymb}
\usepackage{tikz}
\usepackage{pgfplots}
\usepackage{amsmath}
\usepackage{enumitem}

\newcommand{\module}[1]{\|#1\|}

\title{Vectors}
\author{-}
\date{Gener 2023}

\begin{document}

\maketitle

\section{Introduction}
Donats els punts $A$ i $B$ el vector $\vec{AB} = B - A \longrightarrow \vec{AB} = \begin{pmatrix}
    b_x - a_x \\
    b_y - a_y
    \end{pmatrix}$\\
$\Vec{u} = \begin{pmatrix}
    u_1 \\
    u_2
\end{pmatrix}$\\[5pt]
$
\Vec{v} = \begin{pmatrix}
    v_1 \\
    v_2
\end{pmatrix}
$
\\[10pt]
$$\Vec{u} + \Vec{v} = \begin{pmatrix}
    u_1 + v_1 \\
    u_2 + v_2
\end{pmatrix}
$$
$$\Vec{u} - \Vec{v} = \begin{pmatrix}
    u_1 - v_1 \\
    u_2 - v_2
\end{pmatrix}$$
\subsection{Coneixaments Previs}

\subsection{Angle d'un vector}

$$\tan(\alpha)=\frac{u_2}{u_1}$$
$$\alpha=\arctan\left(\frac{u_2}{u_1}\right)$$

\subsubsection{Mutiplicació per un escalar}

    $$
    k \cdot \Vec{u} = \begin{pmatrix}
        k \cdot u_1 \\
        k \cdot u_2
    \end{pmatrix}
    , k \in R
    $$
    

\subsubsection{Mutiplicació de dos vectors - Producte  escalar}

$$\Vec{u} \cdot \Vec{v} = u_1 \cdot v_1 + u_2 + v_2$$
$$\Vec{u} \cdot \vec{v} = \module{\Vec{u}} \cdot \module{\Vec{v}} \cdot \cos\left({\widehat{\Vec{u}, \Vec{v}}}\right)$$
\subsubsection{Vector Unitari}
El mòdul del vector és 1

$$\hat{u} = \dfrac{1}{\module{\Vec{u}}} \cdot \vec{u}$$

\section{Equacions de la recta}

Siguin $A = \left( a_1, a_2 \right)$ i $\Vec{u}=\begin{pmatrix}
    u_1 \\
    u_2
\end{pmatrix}$

\subsection{Equació vectorial}
$$
\begin{pmatrix}
    x \\
    y
\end{pmatrix}
=
\begin{pmatrix}
    a_1 \\
    a_2
\end{pmatrix}
+ \lambda
\begin{pmatrix}
    u_1 \\
    u_2
\end{pmatrix}
$$
$$\Vec{OX} = \Vec{OA} + \lambda \cdot \Vec{u}$$
\subsection{Equació paramètrica}
Dividir l'equació vectorial en dos
\[
\begin{cases}
    x = a_1 + \lambda \cdot u_1 \\
    y = a_2 + \lambda \cdot u_2
\end{cases}\]

\subsection{Equació contínua}
Resultata d'igualar les equacions peramètriques per $\lambda$.
$$
\frac{x-a_1}{u_1} = \frac{y-a_2}{u_2}
$$

\subsection{Equació punt-pendent}
$\dfrac{u_2}{u_1} = m$ on $m$ és el pendent de la recta.
$$
y-a_2=\frac{u_2}{u_1}\cdot \left(x-a_1\right)
$$

\subsection{Equació General o implicita}
$$Ax + By + C = 0$$

$$\begin{pmatrix}
    A \\
    B
\end{pmatrix} \perp \vec{u}$$
\subsection{Equació explícita}

$$y=mx+n$$
$n$ és la ordenada a l'origen, és a dir, valor de $y$ que fa que $x = 0$

\section{Equació canònica}
$$\frac{x}{a}+\frac{y}{b}=1$$
$a$ i $b$ són els punts de tall no nuls
$(a, 0)$ i $(0, b)$

\section{Propietats del producte escalar}

$$\vec{u} \perp \vec{v} \Longleftrightarrow \vec{u} \cdot \vec{v} = 0$$
$$\vec{u} \parallel \vec{v} \Longleftrightarrow |\vec{u} \cdot \vec{v}| = \module{\vec{u}}\cdot\module{\vec{v}}$$
$$\vec{u}\cdot\vec{v} < 0 \Longleftrightarrow 90\degree < \theta < 180\degree $$
$$\vec{u}\cdot\vec{v} > 0 \Longleftrightarrow 0\degree < \theta < 90\degree$$

\section{Segments}
\subsection{Punt mig d'un segment}

Donat el segment $\overline{AB}$ el punt mig $M$ és la suma del punt $A$ més el vector $\vec{AB}$ entre 2
$$\vec{0M} = \vec{0A}+\frac{1}{2}\cdot\vec{AB}$$

\subsection{Partir un segment en parts iguals}

Doant el segment $\overline{PQ}$ amb punts $M$ i $N$,
el vector $\vec{PM}$ és un terç del vector $\vec{PQ}$

$$\vec{0M} = \vec{0P}+ \frac{1}{3} \cdot\vec{PQ}$$
La formula general diria així:

$$\vec{0M} = \vec{0P} + \frac{1}{n} \cdot \vec{PQ}$$
On $\overline{PQ}$ és un segment, $M$ el punt que es vol saber i $n$ el nombre de punts totals.

\section{Posició relativa}
$P(x_0, y_0)$\\
$r: Ax + By + C = 0$ \\[5pt]
$r: \begin{pmatrix}
    x \\
    y   
\end{pmatrix} = \begin{pmatrix}
    a_1 \\
    a_2
\end{pmatrix} + \lambda \begin{pmatrix}
    u_1 \\
    u_2
\end{pmatrix}$ \\[5pt]
$s: A'x + B'y +C = 0$\\[5pt]
$s: \begin{pmatrix}
    x \\
    y   
\end{pmatrix} = \begin{pmatrix}
    b_1 \\
    b_2
\end{pmatrix} + \lambda \begin{pmatrix}
    v_1 \\
    v_2
\end{pmatrix}$ \\
\subsection{Punt - Recta}
Per compropbar si un punt pertany a una recta:
$$P \in r, Ax_0 + By_0 + C = 0$$
$$P \notin r, Ax_0 + By_0 + C \neq 0$$

\subsection{Recta - Recta}

\subsubsection{Paral·leles}

$$ \frac{A}{A'} = \frac{B}{B'} \neq \frac{C}{C'}$$
$\vec{u}, \vec{v}$ són linear dependents, $\vec{u} = k \cdot\vec{v}$\\
Cap punt de $r$ pertany a $s$, $A_r \notin s$\\
$|\vec{u}\cdot\vec{v}| = \module{\vec{u}}\cdot\module{\vec{v}}$
\\[10pt]
\subsubsection{Coincidents}
$$\frac{A}{A'}=\frac{B}{B'}=\frac{C}{C'}$$
$\vec{u}, \vec{v}$ són linealment dependents\\
Tots els punts de $r$ pertanyen a $s$, $A_r \in s$
\\[10pt]
\subsubsection{Secants}
$$\frac{A}{A'}\neq\frac{B}{B'}$$
$\vec{u},\vec{v}$ són linealment independents
\\[10pt]
\subsubsection{Perpendiculars}
$$\vec{u}\cdot\vec{v}=0$$
$$
\begin{cases}
    r: y = m_1 \cdot x + n_1 \\
    s: y = m_2 \cdot x + n_2
\end{cases}
r \perp s \Longleftrightarrow m_1 \cdot m_2 = -1
$$
$$\vec{n_u} = \begin{pmatrix}
    - u_2 \\
    u_1
\end{pmatrix}$$

\section{Rectes singulars}
mediatriu: \textit{recte $\perp$ a una altre recta i que pasa pel seu punt mig.}\\
bisactriu: \textit{recte que parteix l'angle que formen dos altres rectes entre dos.}\\
altura: \textit{recte $\perp$ a un costat d'un triangle que pasa pel vertex oposat.} \\
mediana: \textit{recte que va d'un punt mig d'un costat d'un triangle fins al vertex oposat.}\\

\section{Problemes metrics}
\subsection{Punt - Punt}
Doants els punts $A ( a_1, a_2)$ i $B(b_1, b_2)$ la distancia entre els punts és el modul del vector.
$$d(A, B) = \module{\vec{AB}}
=\sqrt{\left(b_1-a_1\right)^2+\left(b_2-a_2\right)^2}$$
\subsection{Recte - Punt}
Si $P \in r, d(P, r) = 0$\\
Si $P \notin r, d(P, r) = \module{\vec{PQ}}$ on $Q$ és la intersecció entre $r$ i la recta $\perp r$\\
El punt $Q$ és la projecció de $P$ sobre $r$
$$d(P, r)= \frac{|A\cdot x_0+B\cdot y_0+C|}{\sqrt{A^2+B^2}}$$
\subsection{Recta - Recta}
Coincidents, $d(r, s) = 0$\\
Secants, $d(r, s) = 0$\\
Paral·leles, $d(r, s) = d(P_r, s)$
\subsection{Angles}
Coincidents, $\alpha (r, s) = 0$\\
Paral·leles, $\alpha (r, s) = 0$\\
Secants: $$\alpha (r,s) = \arccos \left( \frac{\vec{u}\cdot \vec{v}}{\module{\vec{u}}\cdot\module{\vec{v}}} \right)$$

\subsection{Punt Simètric}
\subsubsection{Simètria respecte un punt}

Sigui $A$ un punt, $B$ un altre punt i $A'$ el simetric de $A$ respecte $B$
$$\vec{0A'} = \vec{0A} + \vec{AB}$$

\subsubsection{Simètria respecte una recte}

Siguin $r: y=m_r\cdot x+n_s$ i $A=(a_1, a_2)$\\[10pt]
1 - Trobar $\perp$ de r que passa per $A$\\[5pt]
Si $r \perp s, m_r \cdot m_s = -1$\\
$$s: y = m_s\cdot x +n_s$$\\[10pt]
2 - Troba $n_s$ substituin el punt $A$ en la equqació \\[5pt]
$$a_2 = m_s \cdot a_1 + n_s$$\\[10pt]
3 - Troba el punt $M$ el cual és la intersecció entre rectes\\[5pt]
$$\begin{cases}
    r: y = m_r \cdot x + n_r \\
    s: y = m_s \cdot x + n_s
\end{cases}
$$
4 - Solucionar el problema com la simetrai de punts
$$\vec{0A'} = \vec{0M} + \vec{AM}$$
\newpage
\subsection{Llocs geomètrics}
\underline{Mediatriu}: \textit{Lloc geomètric dels puns que equidisten 2 punts. És una recta.}\\
\underline{Circumferència}: \textit{Lloc geomètric dels punts que equidisten d'un punt.}\\
Equació d'una circumferència:
$$\left(x-a\right)^2+\left(y-b\right)^2=r^2$$
\begin{align*}
    \begin{tikzpicture}
        \draw[<->] (-1.5, 0) -- (1.5, 0) node[above] {$x$};
        \draw[<->] (0, -1.5) -- (0, 1.5) node[above] {$y$};
        \draw(0, 0) circle (1cm);
    \end{tikzpicture}
\end{align*}
Equació general d'una circumferència:
$$x^2+y^2=Ax+By+C=0$$
$$A=-2a, B=-2b, C=a^2+b^2-r^2$$
\subsubsection{Intersecció entre circumferència i recta}
Siguin $r$ és el radi, $C$ és el centre i $s$ una recta.\\[10pt]
Tangent: 1 intersecció $d(s, C) = r$\\
Secant: 2 interseccions $d(s, C) < r$\\
Exterior: 0 interseccions $d(s, C) > r$\\[5pt]
Les interseccions són les solucions de les equaicons.
\subsubsection{El·lipse}
\textit{Lloc geomètric dels puns, tals que la suma de distancia d'un punt a dos de donats és constant.}\\
$$\frac{\left(x-k\right)^2}{a^2}+\frac{\left(y-h\right)^2}{b^2}=1$$
On $(k, h)$ és el centre de la el·lipse, el punt mig entre els dos focus.\\
$a$ és el semi eix horitzontal i $b$ el semi eix vertical.
\begin{align*}
    \begin{tikzpicture}
        \draw[<->] (-1.5, 0) -- (1.5, 0) node[above] {$x$};
        \draw[<->] (0, -1.5) -- (0, 1.5) node[above] {$y$};
        \draw(0, 0) ellipse (1cm and 0.5cm);
    \end{tikzpicture}
\end{align*}
\newpage
\subsubsection{Paràbola}
\textit{Lloc geomètric dels punts que equidisten d'un punt (focus) i una recta (directriu).}\\
\underline{Directriu horitzontal}
$$y=ax^2+bx+c$$
$$x^2=-2py$$
$p$ és la distancia entre el focus i la directriu.\\

\underline{Directriu vertical}
$$x=ay^2+by+c$$
$$y^2=-2px$$
\subsubsection{Hipèrbola}
\textit{Llog geomètric dels punts tals que la diferència de distáncies d'un punt a dos donats (focus) és constant.}\\
$$\frac{\left(x-k\right)^2}{a^2}-\frac{\left(y-h\right)^2}{b^2}=1$$

\end{document}
