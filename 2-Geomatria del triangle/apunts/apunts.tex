\documentclass[12pt,a4paper]{article}
\usepackage{gensymb}
\usepackage[left=2cm,right=2cm,top=2cm,bottom=2cm]{geometry}
\usepackage{amsfonts}
\usepackage{amsmath}
\usepackage{enumitem}
\usepackage{tikz}
\usepackage{pgfplots}

\newcommand{\reals}{\mathbb{R}}
\newcommand{\naturals}{\mathbb{Z}}
\newcommand{\module}[1]{\|#1\|}
\newcommand{\complex}{\mathbb{C}}
\newcommand{\drawgraph}{ \draw[-] (-1, 0) -- (1, 0) node[right] {$x$};
\draw[-] (0, -1) -- (0, 1) node[above] {$y$};}

\title{La geometria del triangle}
\author{-}
\date{Febrer 2023}

\pgfplotsset{compat=1.18}
\begin{document}

\maketitle
\section{Propietats dels traingles}

$\sum angles = 180\degree = \pi$ $rad$
La suma dels catets és més gran que la hipotenusa.\\
En un angle agut:
\begin{enumerate}[label= ]
    \item $\sin\theta=\frac{CO}{H}$
    \item $\cos\theta=\frac{CA}{H}$
    \item $\tan\theta=\frac{CO}{CA}$
\end{enumerate}
Propietat fonamental de la trigonometria:
$$\cos^2(\alpha)+\sin^2(\alpha)=1$$
$$\frac{\sin(\alpha)}{\cos(\alpha)}=\tan(\alpha)$$
\section{Cicrumferencia trigonometrica}
És divideix la cicrumferencia en 4 quadrants.\\
Els eixos representens $\cos\alpha$ i $\sin\alpha$.\\
$-1\leq\cos(\alpha)\leq 1$ \\ $-1\leq\sin(\alpha)\leq 1$ 
\subsection{Funcions reciporques}
$$\sin(x)\rightarrow\frac{1}{\sin(x)}=\csc(x)$$
$$\cos(x)\rightarrow\frac{1}{\cos(x)}=\sec(x)$$
$$\tan(x)\rightarrow\frac{1}{\tan(x)}=\cot(x)$$
\subsection{Angle II quadrant}
$\alpha + \beta = \pi$, $\alpha$ i $\beta$ són suplementaris.
$$\sin \beta = \sin(\pi - \beta)$$
$$\cos \beta = -\cos(\pi-\beta)$$
$$\tan \beta = -\tan(\pi-\beta)$$
\subsection{Angle III quadrant}
$$-\sin \beta = \sin(\pi + \beta)$$
$$-\cos \beta = -\cos(\pi+\beta)$$
$$\tan \beta = \tan(\pi+\beta)$$
\subsection{Angle IV quadrant}
$$-\sin \beta = \sin(2\pi - \beta)$$
$$\cos \beta = \cos(2\pi-\beta)$$
$$-\tan \beta = \tan(2\pi-\beta)$$
\subsection{Angles coplementaris}
La suma dels angles és $\frac{\pi}{2}$
$$\sin \beta = \sin(2\pi - \beta)$$
$$\cos \beta = -\cos(2\pi-\beta)$$
$$\tan \beta = \cot(\pi-\beta)$$
\section{Equacions trigonomètriques}
$$\sin x = \frac{1}{2}$$
$$x=\arcsin\left(\frac{1}{2}\right)=\frac{\pi}{6} \rightarrow \frac{\pi}{6}+2\pi k, k \in \naturals$$
$$=\frac{5\pi}{6}\rightarrow \frac{5\pi}{6}+2\pi k, k \in \naturals$$
\section{Area d'un triangle}
$$A_{\triangle}=\frac{1}{2}\cdot b \cdot h$$
$$A_c=\frac{1}{2}\cdot a \cdot b \cdot \sin C$$
$$A_b=\frac{1}{2}\cdot a \cdot c \cdot \sin B$$
$$A_a=\frac{1}{2}\cdot b \cdot c \cdot \sin A$$
\subsection{Teorema del sinus}
$$\frac{\sin C}{c}=\frac{sin B}{b}=\frac{\sin A}{a}$$
\subsection{Teorema del cosinus}
$$a^2+=b^2+c^2-2bc\cdot \cos A$$
$$b^2+=a^2+c^2-2ac\cdot \cos B$$
$$c^2+=a^2+b^2-2ab\cdot \cos C$$

\end{document}
